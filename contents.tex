\section{数据结构}
\subsection{树状数组}
\subsubsection{树状数组}
\input{src/数据结构/树状数组/树状数组.tex}
\lstinputlisting[style=cppstyle]{src/数据结构/树状数组/树状数组.cpp}
\subsubsection{树状数组2}
\input{src/数据结构/树状数组/树状数组2.tex}
\lstinputlisting[style=cppstyle]{src/数据结构/树状数组/树状数组2.cpp}
\subsection{线段树}
\subsubsection{线段树simple}
\lstinputlisting[style=cppstyle]{src/数据结构/线段树/线段树simple.cpp}
\subsubsection{线段树}
\input{src/数据结构/线段树/线段树.tex}
\lstinputlisting[style=cppstyle]{src/数据结构/线段树/线段树.cpp}
\subsubsection{动态开点线段树}
\lstinputlisting[style=cppstyle]{src/数据结构/线段树/动态开点线段树.cpp}
\subsubsection{主席树}
\input{src/数据结构/线段树/主席树.tex}
\lstinputlisting[style=cppstyle]{src/数据结构/线段树/主席树.cpp}
\subsubsection{标记永久化主席树}
\input{src/数据结构/线段树/标记永久化主席树.tex}
\lstinputlisting[style=cppstyle]{src/数据结构/线段树/标记永久化主席树.cpp}
\subsubsection{线段树优化建图}
\input{src/数据结构/线段树/线段树优化建图.tex}
\lstinputlisting[style=cppstyle]{src/数据结构/线段树/线段树优化建图.cpp}
\subsection{ST表}
\input{src/数据结构/ST表.tex}
\lstinputlisting[style=cppstyle]{src/数据结构/ST表.cpp}
\subsection{并查集}
\input{src/数据结构/并查集.tex}
\lstinputlisting[style=cppstyle]{src/数据结构/并查集.cpp}
\subsection{可撤销并查集}
\lstinputlisting[style=cppstyle]{src/数据结构/可撤销并查集.cpp}
\subsection{带权并查集}
\textbf{模板题:}\href{https://www.luogu.com.cn/problem/P1196}{Luogu P1196}

\lstinputlisting[style=cppstyle]{src/数据结构/带权并查集.cpp}
\subsection{字典树}
\lstinputlisting[style=cppstyle]{src/数据结构/字典树.cpp}
\subsection{左偏树}
\lstinputlisting[style=cppstyle]{src/数据结构/左偏树.cpp}
\subsection{智慧集}
\textbf{时间复杂度:}\\
\textbullet{插入:$O(\log(n))$}\\
\textbullet{删除:$O(\log(n))$}\\
\textbullet{第k小:$O(1)$ 前提:每次操作k变化不大}\\
\textbf{空间复杂度:}$O(n)$\\
\textbf{用途:}双指针中位数\\
\textbf{模板题:}\href{https://qoj.ac/contest/1472/problem/7904}{2023ICPC-Jianing-Regional K. Rainbow Subarray}

\lstinputlisting[style=cppstyle]{src/数据结构/智慧集.cpp}
\subsection{欧拉序}
\lstinputlisting[style=cppstyle]{src/数据结构/欧拉序.cpp}
\subsection{波纹疾走树}
\input{src/数据结构/波纹疾走树.tex}
\lstinputlisting[style=cppstyle]{src/数据结构/波纹疾走树.cpp}
\subsection{Splay}
\input{src/数据结构/Splay.tex}
\lstinputlisting[style=cppstyle]{src/数据结构/Splay.cpp}
\section{图论}
\subsection{图的连通性}
\subsubsection{Tarjan割点}
\input{src/图论/图的连通性/Tarjan割点.tex}
\lstinputlisting[style=cppstyle]{src/图论/图的连通性/Tarjan割点.cpp}
\subsubsection{Tarjan割边}
\input{src/图论/图的连通性/Tarjan割边.tex}
\lstinputlisting[style=cppstyle]{src/图论/图的连通性/Tarjan割边.cpp}
\subsubsection{Tarjan强连通分量}
\input{src/图论/图的连通性/Tarjan强连通分量.tex}
\lstinputlisting[style=cppstyle]{src/图论/图的连通性/Tarjan强连通分量.cpp}
\subsubsection{Tarjan点双连通分量}
\input{src/图论/图的连通性/Tarjan点双连通分量.tex}
\lstinputlisting[style=cppstyle]{src/图论/图的连通性/Tarjan点双连通分量.cpp}
\subsubsection{Tarjan边双连通分量}
\textbf{用途:}求边双连通分量\\
\textbf{模板题:}\href{https://www.luogu.com.cn/problem/P8436}{洛谷 P8436}

\lstinputlisting[style=cppstyle]{src/图论/图的连通性/Tarjan边双连通分量.cpp}
\subsection{重链剖分}
\input{src/图论/重链剖分.tex}
\lstinputlisting[style=cppstyle]{src/图论/重链剖分.cpp}
\subsection{长链剖分}
\lstinputlisting[style=cppstyle]{src/图论/长链剖分.cpp}
\subsection{树的重心}
\input{src/图论/树的重心.tex}
\lstinputlisting[style=cppstyle]{src/图论/树的重心.cpp}
\subsection{欧拉回路}
\lstinputlisting[style=cppstyle]{src/图论/欧拉回路.cpp}
\subsection{流和匹配}
\subsubsection{EdmondsKarp}
\textbf{时间复杂度:} $O(|V||E|^2)$ 实际情况一般远低于此复杂度\\
\textbf{空间复杂度:} $O(|V| + |E|)$ \\
\textbf{用途:}求最大流\\
\textbf{模板题:}\href{https://www.luogu.com.cn/problem/P3376}{洛谷 P3376}

\lstinputlisting[style=cppstyle]{src/图论/流和匹配/EdmondsKarp.cpp}
\subsubsection{二分图判定}
\textbf{时间复杂度:} $O(|V| + |E|)$ \\
\textbf{空间复杂度:} $O(|V|)$ \\
\textbf{模板题:}\href{https://www.luogu.com.cn/problem/P1330}{Luogu P1330}
\lstinputlisting[style=cppstyle]{src/图论/流和匹配/二分图判定.cpp}
\subsubsection{二分图最大匹配}
\textbf{时间复杂度:} $O(|V_1||V_2|)$ \\
\textbf{空间复杂度:} $O(|E| + |V_1| + |V_2|)$ \\
\textbf{模板题:}\href{https://www.luogu.com.cn/problem/P3386}{Luogu P3386}
\lstinputlisting[style=cppstyle]{src/图论/流和匹配/二分图最大匹配.cpp}
\section{字符串}
\subsection{KMP}
\input{src/字符串/KMP.tex}
\lstinputlisting[style=cppstyle]{src/字符串/KMP.cpp}
\subsection{EXKMP}
\input{src/字符串/EXKMP.tex}
\lstinputlisting[style=cppstyle]{src/字符串/EXKMP.cpp}
\subsection{字符串哈希}
\input{src/字符串/字符串哈希.tex}
\lstinputlisting[style=cppstyle]{src/字符串/字符串哈希.cpp}
\subsection{马拉车}
\input{src/字符串/马拉车.tex}
\lstinputlisting[style=cppstyle]{src/字符串/马拉车.cpp}
\section{数学}
\subsection{组合数学}
\subsubsection{组合数}
\lstinputlisting[style=cppstyle]{src/数学/组合数学/组合数.cpp}
\subsubsection{卢卡斯定理}
\input{src/数学/组合数学/卢卡斯定理.tex}
\lstinputlisting[style=cppstyle]{src/数学/组合数学/卢卡斯定理.cpp}
\subsection{数论}
\subsubsection{线性筛}
\lstinputlisting[style=cppstyle]{src/数学/数论/线性筛.cpp}
\subsubsection{区间筛}
\input{src/数学/数论/区间筛.tex}
\lstinputlisting[style=cppstyle]{src/数学/数论/区间筛.cpp}
\subsubsection{MillerRabin}
\textbf{模板题:}\href{https://www.luogu.com.cn/problem/SP288}{Luogu SP288}
\textbf{时间复杂度:} $O(k \log^3(n)), k = 7$
\lstinputlisting[style=cppstyle]{src/数学/数论/MillerRabin.cpp}
\subsubsection{PollardRho}
\textbf{模板题:}\href{https://judge.yosupo.jp/problem/factorize}{Factorize}\\
\textbf{时间复杂度:} $O(n ^ {\frac{1}{4}})$

\lstinputlisting[style=cppstyle]{src/数学/数论/PollardRho.cpp}
\subsubsection{中国剩余定理}
\lstinputlisting[style=cppstyle]{src/数学/数论/中国剩余定理.cpp}
\subsubsection{扩展中国剩余定理}
\lstinputlisting[style=cppstyle]{src/数学/数论/扩展中国剩余定理.cpp}
\subsection{线性代数}
\subsubsection{矩阵}
\lstinputlisting[style=cppstyle]{src/数学/线性代数/矩阵.cpp}
\subsubsection{线性基}
\lstinputlisting[style=cppstyle]{src/数学/线性代数/线性基.cpp}
\subsection{多项式}
\subsubsection{FFT}
\textbf{模板题:}\href{https://www.luogu.com.cn/problem/P3803}{Luogu P3803}
\textbf{模板题:}\href{https://atcoder.jp/contests/abc392/tasks/abc392_g}{ABC 392G}
\lstinputlisting[style=cppstyle]{src/数学/多项式/FFT.cpp}
\subsubsection{NTT}
\lstinputlisting[style=cppstyle]{src/数学/多项式/NTT.cpp}
\subsubsection{Poly}
\lstinputlisting[style=cppstyle]{src/数学/多项式/Poly.cpp}
\section{计算几何}
\subsection{凸包}
\input{src/计算几何/凸包.tex}
\lstinputlisting[style=cppstyle]{src/计算几何/凸包.cpp}
\section{杂项}
\subsection{动态bitset}
\lstinputlisting[style=cppstyle]{src/杂项/动态bitset.cpp}
\subsection{取整}
\lstinputlisting[style=cppstyle]{src/杂项/取整.cpp}
\subsection{康托展开}
\input{src/杂项/康托展开.tex}
\lstinputlisting[style=cppstyle]{src/杂项/康托展开.cpp}
\subsection{逆康托展开}
\input{src/杂项/逆康托展开.tex}
\lstinputlisting[style=cppstyle]{src/杂项/逆康托展开.cpp}
\subsection{日期公式}
\lstinputlisting[style=cppstyle]{src/杂项/日期公式.cpp}
\subsection{高精度}
\input{src/杂项/高精度.tex}
\lstinputlisting[style=cppstyle]{src/杂项/高精度.cpp}
\subsection{高维前缀和}
\input{src/杂项/高维前缀和.tex}
\lstinputlisting[style=cppstyle]{src/杂项/高维前缀和.cpp}
\subsection{命令行}
\lstinputlisting[style=cppstyle]{src/杂项/命令行.cpp}
\section{附录}
\subsection{编译参数}
\input{src/附录/编译参数.tex}
\subsection{OJ测试}
\input{src/附录/OJ测试.tex}
\subsection{组合数学公式}
\input{src/附录/组合数学公式.tex}
\subsection{随机素数}
\input{src/附录/随机素数.tex}
\subsection{常数表}
\input{src/附录/常数表.tex}