\section{数据结构}
\subsection{树状数组}
\subsubsection{树状数组}
\input{src/数据结构/树状数组/树状数组.tex}
\lstinputlisting[style=cppstyle]{src/数据结构/树状数组/树状数组.cpp}
\subsubsection{树状数组2}
\input{src/数据结构/树状数组/树状数组2.tex}
\lstinputlisting[style=cppstyle]{src/数据结构/树状数组/树状数组2.cpp}
\subsection{线段树}
\subsubsection{线段树simple}
\lstinputlisting[style=cppstyle]{src/数据结构/线段树/线段树simple.cpp}
\subsubsection{线段树}
\input{src/数据结构/线段树/线段树.tex}
\lstinputlisting[style=cppstyle]{src/数据结构/线段树/线段树.cpp}
\subsubsection{动态开点线段树}
\lstinputlisting[style=cppstyle]{src/数据结构/线段树/动态开点线段树.cpp}
\subsubsection{主席树}
\input{src/数据结构/线段树/主席树.tex}
\lstinputlisting[style=cppstyle]{src/数据结构/线段树/主席树.cpp}
\subsubsection{标记永久化主席树}
\input{src/数据结构/线段树/标记永久化主席树.tex}
\lstinputlisting[style=cppstyle]{src/数据结构/线段树/标记永久化主席树.cpp}
\subsubsection{线段树优化建图}
\input{src/数据结构/线段树/线段树优化建图.tex}
\lstinputlisting[style=cppstyle]{src/数据结构/线段树/线段树优化建图.cpp}
\subsection{ST表}
\textbf{时间复杂度:}\\
\textbullet{初始化:$O(n\log(n))$}\\
\textbullet{查询:$O(1)$}\\
\textbf{空间复杂度:}$O(nlog(n))$\\
\textbf{用途:}RMQ问题,不支持修改\\
\textbf{模板题:}\href{https://www.luogu.com.cn/problem/P3865}{Luogu P3865}
\lstinputlisting[style=cppstyle]{src/数据结构/ST表.cpp}
\subsection{并查集}
\input{src/数据结构/并查集.tex}
\lstinputlisting[style=cppstyle]{src/数据结构/并查集.cpp}
\subsection{可撤销并查集}
\lstinputlisting[style=cppstyle]{src/数据结构/可撤销并查集.cpp}
\subsection{带权并查集}
\textbf{模板题:}\href{https://www.luogu.com.cn/problem/P1196}{Luogu P1196}

\lstinputlisting[style=cppstyle]{src/数据结构/带权并查集.cpp}
\subsection{字典树}
\lstinputlisting[style=cppstyle]{src/数据结构/字典树.cpp}
\subsection{左偏树}
\lstinputlisting[style=cppstyle]{src/数据结构/左偏树.cpp}
\subsection{智慧集}
\textbf{时间复杂度:}\\
\textbullet{插入:$O(\log(n))$}\\
\textbullet{删除:$O(\log(n))$}\\
\textbullet{第k小:$O(1)$ 前提:每次操作k变化不大}\\
\textbf{空间复杂度:}$O(n)$\\
\textbf{用途:}双指针中位数\\
\textbf{模板题:}\href{https://qoj.ac/contest/1472/problem/7904}{2023ICPC-Jinan-Regional K. Rainbow Subarray}

\lstinputlisting[style=cppstyle]{src/数据结构/智慧集.cpp}
\subsection{欧拉序}
\lstinputlisting[style=cppstyle]{src/数据结构/欧拉序.cpp}
\subsection{波纹疾走树}
\input{src/数据结构/波纹疾走树.tex}
\lstinputlisting[style=cppstyle]{src/数据结构/波纹疾走树.cpp}
\subsection{Splay}
\input{src/数据结构/Splay.tex}
\lstinputlisting[style=cppstyle]{src/数据结构/Splay.cpp}
\section{图论}
\subsection{图的连通性}
\subsubsection{Tarjan割点}
\input{src/图论/图的连通性/Tarjan割点.tex}
\lstinputlisting[style=cppstyle]{src/图论/图的连通性/Tarjan割点.cpp}
\subsubsection{Tarjan割边}
\input{src/图论/图的连通性/Tarjan割边.tex}
\lstinputlisting[style=cppstyle]{src/图论/图的连通性/Tarjan割边.cpp}
\subsubsection{Tarjan强连通分量}
\input{src/图论/图的连通性/Tarjan强连通分量.tex}
\lstinputlisting[style=cppstyle]{src/图论/图的连通性/Tarjan强连通分量.cpp}
\subsubsection{Tarjan点双连通分量}
\input{src/图论/图的连通性/Tarjan点双连通分量.tex}
\lstinputlisting[style=cppstyle]{src/图论/图的连通性/Tarjan点双连通分量.cpp}
\subsubsection{Tarjan边双连通分量}
\textbf{用途:}求边双连通分量\\
\textbf{模板题:}\href{https://www.luogu.com.cn/problem/P8436}{洛谷 P8436}

\lstinputlisting[style=cppstyle]{src/图论/图的连通性/Tarjan边双连通分量.cpp}
\subsection{重链剖分}
\input{src/图论/重链剖分.tex}
\lstinputlisting[style=cppstyle]{src/图论/重链剖分.cpp}
\subsection{长链剖分}
\lstinputlisting[style=cppstyle]{src/图论/长链剖分.cpp}
\subsection{树的重心}
\textbf{定义:}如果在树中选择某个节点并删除,这棵树将分为若干棵子树,统计子树节点数并记录最大值。取遍树上所有节点,使此最大值取到最小的节点被称为整个树的重心。

\lstinputlisting[style=cppstyle]{src/图论/树的重心.cpp}
\subsection{欧拉回路}
\lstinputlisting[style=cppstyle]{src/图论/欧拉回路.cpp}
\subsection{流和匹配}
\subsubsection{EdmondsKarp}
\textbf{时间复杂度:} $O(|V||E|^2)$ 实际情况一般远低于此复杂度\\
\textbf{空间复杂度:} $O(|V| + |E|)$ \\
\textbf{用途:}求最大流\\
\textbf{模板题:}\href{https://www.luogu.com.cn/problem/P3376}{洛谷 P3376}

\lstinputlisting[style=cppstyle]{src/图论/流和匹配/EdmondsKarp.cpp}
\subsubsection{二分图判定}
\textbf{时间复杂度:} $O(|V| + |E|)$ \\
\textbf{空间复杂度:} $O(|V|)$ \\
\textbf{模板题:}\href{https://www.luogu.com.cn/problem/P1330}{Luogu P1330}
\lstinputlisting[style=cppstyle]{src/图论/流和匹配/二分图判定.cpp}
\subsubsection{二分图最大匹配}
\textbf{时间复杂度:} $O(|V_1||V_2|)$ \\
\textbf{空间复杂度:} $O(|E| + |V_1| + |V_2|)$ \\
\textbf{模板题:}\href{https://www.luogu.com.cn/problem/P3386}{Luogu P3386}
\lstinputlisting[style=cppstyle]{src/图论/流和匹配/二分图最大匹配.cpp}
\section{字符串}
\subsection{KMP}
\input{src/字符串/KMP.tex}
\lstinputlisting[style=cppstyle]{src/字符串/KMP.cpp}
\subsection{EXKMP}
\input{src/字符串/EXKMP.tex}
\lstinputlisting[style=cppstyle]{src/字符串/EXKMP.cpp}
\subsection{字符串哈希}
\input{src/字符串/字符串哈希.tex}
\lstinputlisting[style=cppstyle]{src/字符串/字符串哈希.cpp}
\subsection{马拉车}
\input{src/字符串/马拉车.tex}
\lstinputlisting[style=cppstyle]{src/字符串/马拉车.cpp}
\section{数学}
\subsection{组合数学}
\subsubsection{组合数}
\lstinputlisting[style=cppstyle]{src/数学/组合数学/组合数.cpp}
\subsubsection{卢卡斯定理}
\small {
$C_n^m \pmod p = C_{\lfloor \frac{n}{p} \rfloor}^{\lfloor \frac{m}{p} \rfloor} * C_{n \bmod p}^{m \bmod p}$
}

\lstinputlisting[style=cppstyle]{src/数学/组合数学/卢卡斯定理.cpp}
\subsection{数论}
\subsubsection{线性筛}
\lstinputlisting[style=cppstyle]{src/数学/数论/线性筛.cpp}
\subsubsection{区间筛}
\input{src/数学/数论/区间筛.tex}
\lstinputlisting[style=cppstyle]{src/数学/数论/区间筛.cpp}
\subsubsection{MillerRabin}
\textbf{模板题:}\href{https://www.luogu.com.cn/problem/SP288}{Luogu SP288}\\
\textbf{时间复杂度:} $O(k \log^3(n)), k = 7$
\lstinputlisting[style=cppstyle]{src/数学/数论/MillerRabin.cpp}
\subsubsection{PollardRho}
\textbf{模板题:}\href{https://judge.yosupo.jp/problem/factorize}{Factorize}\\
\textbf{时间复杂度:} $O(n ^ {\frac{1}{4}})$

\lstinputlisting[style=cppstyle]{src/数学/数论/PollardRho.cpp}
\subsubsection{中国剩余定理}
\lstinputlisting[style=cppstyle]{src/数学/数论/中国剩余定理.cpp}
\subsubsection{扩展中国剩余定理}
\lstinputlisting[style=cppstyle]{src/数学/数论/扩展中国剩余定理.cpp}
\subsection{线性代数}
\subsubsection{矩阵}
\lstinputlisting[style=cppstyle]{src/数学/线性代数/矩阵.cpp}
\subsubsection{线性基}
\lstinputlisting[style=cppstyle]{src/数学/线性代数/线性基.cpp}
\subsection{多项式}
\subsubsection{FFT}
\textbf{模板题:}\href{https://www.luogu.com.cn/problem/P3803}{Luogu P3803}
\textbf{模板题:}\href{https://atcoder.jp/contests/abc392/tasks/abc392_g}{ABC 392G}
\lstinputlisting[style=cppstyle]{src/数学/多项式/FFT.cpp}
\subsubsection{NTT}
\lstinputlisting[style=cppstyle]{src/数学/多项式/NTT.cpp}
\subsubsection{Poly}
\lstinputlisting[style=cppstyle]{src/数学/多项式/Poly.cpp}
\section{计算几何}
\subsection{凸包}
\input{src/计算几何/凸包.tex}
\lstinputlisting[style=cppstyle]{src/计算几何/凸包.cpp}
\section{杂项}
\subsection{动态bitset}
\lstinputlisting[style=cppstyle]{src/杂项/动态bitset.cpp}
\subsection{取整}
\lstinputlisting[style=cppstyle]{src/杂项/取整.cpp}
\subsection{康托展开}
\input{src/杂项/康托展开.tex}
\lstinputlisting[style=cppstyle]{src/杂项/康托展开.cpp}
\subsection{逆康托展开}
\input{src/杂项/逆康托展开.tex}
\lstinputlisting[style=cppstyle]{src/杂项/逆康托展开.cpp}
\subsection{日期公式}
\lstinputlisting[style=cppstyle]{src/杂项/日期公式.cpp}
\subsection{高精度}
\input{src/杂项/高精度.tex}
\lstinputlisting[style=cppstyle]{src/杂项/高精度.cpp}
\subsection{高维前缀和}
\small {
\textbf{时间复杂度:}$O(n2^n)$\\
\textbf{空间复杂度:}$O(n2^n)$\\
\textbf{用途:}位集合中,求出某个集合的所有子集值之和以及其他可加性操作\\
\textbf{模板题:}\href{https://atcoder.jp/contests/arc100/tasks/arc100_c}{AtCoder ARC100 C}
}
\lstinputlisting[style=cppstyle]{src/杂项/高维前缀和.cpp}
\subsection{命令行}
\lstinputlisting[style=cppstyle]{src/杂项/命令行.cpp}
\section{附录}
\subsection{编译参数}
-D\_GLIBCXX\_DEBUG : STL debugmode\\
-fsanitize=address :内存错误检查\\
-fsanitize=undefined :UB检查\\
\subsection{OJ测试}
\begin{lstlisting}[style=cppstyle]
#include <bits/stdc++.h>
using i64 = long long;

int main() {
    std::ios::sync_with_stdio(false);
    std::cin.tie(nullptr);
    volatile unsigned j = 1;
    for(unsigned i = 1; i <= (unsigned)65E7; i++) {
        j = 0;
        j += i;
        j *= i;
        j /= i;
    }
    return 0;
}
\end{lstlisting}
\begin{center}
    \begin{tabular}{|r|r|r|r|r|r|}
        \hline
        \rowcolor{gray!20}
        Local       & QOJ       & AtCoder   & LuoGu     & Codeforces    & Nowcoder  \\ \hline
        1000ms      & 1090ms    & 1120ms    & 1572ms    & 1718ms        & 18070ms   \\ \hline
    \end{tabular}\\
\end{center}

\subsection{组合数学公式}
\textbf{性质1:}\\[8pt]
{\large $\mathrm{C}_{\mathrm{n}}^{\mathrm{m}} = \mathrm{C}_{\mathrm{n}}^{\mathrm{n} - \mathrm{m}}$}\\[12pt]
\textbf{性质2:}\\[8pt]
{\large $\mathrm{C}_{\mathrm{n} + \mathrm{m} + 1}^{\mathrm{m}} = \sum_{i = 0}^{\mathrm{m}} \mathrm{C}_{\mathrm{n} + i}^{i}$}\\[12pt]
\textbf{性质3:}\\[8pt]
{\large $\mathrm{C}_{\mathrm{n}}^{\mathrm{m}} \cdot \mathrm{C}_{\mathrm{m}}^{\mathrm{r}} = \mathrm{C}_{\mathrm{n}}^{\mathrm{r}} \cdot \mathrm{C}_{\mathrm{n} - \mathrm{r}}^{\mathrm{m} - \mathrm{r}}$}\\[12pt]
\textbf{性质4(二项式定理):}\\[8pt]
{\large $\sum_{i = 0}^{\mathrm{n}} \left( \mathrm{C}_{\mathrm{n}}^{i} \cdot x^{i} \right) = (1 + x)^{\mathrm{n}}$}\\[12pt]
{\large $\sum_{i = 0}^{\mathrm{n}} \mathrm{C}_{\mathrm{n}}^{i} = 2^{\mathrm{n}}$}\\[12pt]
\textbf{性质5:}\\[8pt]
{\large $\sum_{i = 0}^{\mathrm{n}} \left( (-1)^{i} \cdot \mathrm{C}_{\mathrm{n}}^{i} \right) = 0$}\\[12pt]
\textbf{性质6:}\\[8pt]
{\large $\mathrm{C}_{\mathrm{n}}^{0} + \mathrm{C}_{\mathrm{n}}^{2} + \cdots = \mathrm{C}_{\mathrm{n}}^{1} + \mathrm{C}_{\mathrm{n}}^{3} + \cdots = 2^{\mathrm{n} - 1}$}\\[12pt]
\textbf{性质7:}\\[8pt]
{\large $\mathrm{C}_{\mathrm{n} + \mathrm{m}}^{\mathrm{r}} = \sum_{i = 0}^{\min(\mathrm{n}, \mathrm{m}, \mathrm{r})} \left( \mathrm{C}_{\mathrm{n}}^{i} \cdot \mathrm{C}_{\mathrm{m}}^{\mathrm{r} - i} \right)$}\\[12pt]
{\large $\mathrm{C}_{\mathrm{n} + \mathrm{m}}^{\mathrm{n}} = \mathrm{C}_{\mathrm{n} + \mathrm{m}}^{\mathrm{m}} = \sum_{i = 0}^{\min(\mathrm{n}, \mathrm{m})} \left( \mathrm{C}_{\mathrm{n}}^{i} \cdot \mathrm{C}_{\mathrm{m}}^{i} \right), \quad (\mathrm{r} = \mathrm{n}\ |\ \mathrm{r} = \mathrm{m})$}\\[12pt]
\textbf{性质8:}\\[8pt]
{\large $\mathrm{m} \cdot \mathrm{C}_{\mathrm{n}}^{\mathrm{m}} = \mathrm{n} \cdot \mathrm{C}_{\mathrm{n} - 1}^{\mathrm{m} - 1}$}\\[12pt]
\textbf{性质9:}\\[8pt]
{\large $\sum_{i = 0}^{\mathrm{n}} \left( \mathrm{C}_{\mathrm{n}}^{i} \cdot i^2 \right) = \mathrm{n}(\mathrm{n} + 1) \cdot 2^{\mathrm{n} - 2}$}\\[12pt]
\textbf{性质10:}\\[8pt]
{\large $\sum_{i = 0}^{\mathrm{n}} \left( \mathrm{C}_{\mathrm{n}}^{i} \right)^2 = \mathrm{C}_{2\mathrm{n}}^{\mathrm{n}}$}\\[12pt]
\subsection{随机素数}
979345007 986854057502126921\\
935359631 949054338673679153\\
931936021 989518940305146613\\
984974633 972090414870546877\\
984858209 956380060632801307\\

\subsection{常数表}
\begin{center}
    \begin{tabular}{|r|r|r|r|r|r|}
        \hline
        \rowcolor{gray!20}
        $n$         & $\log_{10}n$  & $n!$          & $C(n,n/2)$    & $\mathrm{LCM}(1...n)$ & $P_n$         \\ \hline
        2           & 0.30102999    & 2             & 2             & 2                     & 2             \\ \hline
        3           & 0.47712125    & 6             & 3             & 6                     & 3             \\ \hline
        4           & 0.60205999    & 24            & 6             & 12                    & 5             \\ \hline
        5           & 0.69897000    & 120           & 10            & 60                    & 7             \\ \hline
        6           & 0.77815125    & 720           & 20            & 60                    & 11            \\ \hline
        7           & 0.84509804    & 5040          & 35            & 420                   & 15            \\ \hline
        8           & 0.90308998    & 40320         & 70            & 840                   & 22            \\ \hline
        9           & 0.95424251    & 362880        & 126           & 2520                  & 30            \\ \hline
        10          & 1.00000000    & 3628800       & 252           & 2520                  & 42            \\ \hline
        11          & 1.04139269    & 39916800      & 462           & 27720                 & 56            \\ \hline
        12          & 1.07918125    & 479001600     & 924           & 27720                 & 77            \\ \hline
        15          & 1.17609126    & 1.31e12       & 6435          & 360360                & 176           \\ \hline
        20          & 1.30103000    & 2.43e18       & 184756        & 232792560             & 627           \\ \hline
        25          & 1.39794001    & 1.55e25       & 5200300       & 26771144400           & 1958          \\ \hline
        30          & 1.47712125    & 2.65e32       & 155117520     & 1.444e14              & 5604          \\ \hline
        $P_n$       & $37338_{40}$  & $204226_{50}$ & $966467_{60}$ & $190569292_{100}$     & $1e9_{114}$   \\ \hline
    \end{tabular}\\
\end{center}
\begin{center}
    $\max\omega(n)$:小于等于n中的数最大质因数个数\\
    $\max d(n)$:小于等于n中的数最大因数个数\\
    $\pi(n)$:小于等于n中的数最大互质数个数\\
    \begin{tabular}{|r|r|r|r|r|r|r|}
        \hline
        \rowcolor{gray!20}
        $n\leq$         & 10        & 100       & 1e3       & 1e4       & 1e5       & 1e6           \\ \hline
        $\max\omega(n)$ & 2         & 3         & 4         & 5         & 6         & 7             \\ \hline
        $\max d(n)$     & 4         & 12        & 32        & 64        & 128       & 240           \\ \hline
        $\pi(n)$        & 4         & 25        & 168       & 1229      & 9592      & 78498         \\ \hline
        \rowcolor{gray!20}
        $n\leq$         & 1e7       & 1e8       & 1e9       & 1e10      & 1e11      & 1e12          \\ \hline
        $\max\omega(n)$ & 8         & 8         & 9         & 10        & 10        & 11            \\ \hline
        $\max d(n)$     & 448       & 768       & 1344      & 2304      & 4032      & 6720          \\ \hline
        $\pi(n)$        & 664579    & 5761455   & 5.08e7    & 4.55e8    & 4.12e9    & 3.7e10        \\ \hline
        \rowcolor{gray!20}
        $n\leq$         & 1e13      & 1e14      & 1e15      & 1e16      & 1e17      & 1e18          \\ \hline
        $\max\omega(n)$ & 12        & 12        & 13        & 13        & 14        & 15            \\ \hline
        $\max d(n)$     & 10752     & 17280     & 26880     & 41472     & 64512     & 103680        \\ \hline
        $\pi(n)$        & \multicolumn{5}{l}{Prime number theorem: $\pi(x) \sim \frac{x}{\log(x)}$}&\\ \hline
    \end{tabular}
\end{center}