\section{编译参数}
-D\_GLIBCXX\_DEBUG : STL debugmode\\
-fsanitize=address :内存错误检查\\
-fsanitize=undefined :UB检查
\section{随机素数}
979345007 986854057502126921\\
935359631 949054338673679153\\
931936021 989518940305146613\\
984974633 972090414870546877\\
984858209 956380060632801307\\

\section{常用组合数学公式}
\textbf{性质1:}\\[8pt]
{\large $\mathrm{C}_{\mathrm{n}}^{\mathrm{m}} = \mathrm{C}_{\mathrm{n}}^{\mathrm{n} - \mathrm{m}}$}\\[12pt]
\textbf{性质2:}\\[8pt]
{\large $\mathrm{C}_{\mathrm{n} + \mathrm{m} + 1}^{\mathrm{m}} = \sum_{i = 0}^{\mathrm{m}} \mathrm{C}_{\mathrm{n} + i}^{i}$}\\[12pt]
\textbf{性质3:}\\[8pt]
{\large $\mathrm{C}_{\mathrm{n}}^{\mathrm{m}} \cdot \mathrm{C}_{\mathrm{m}}^{\mathrm{r}} = \mathrm{C}_{\mathrm{n}}^{\mathrm{r}} \cdot \mathrm{C}_{\mathrm{n} - \mathrm{r}}^{\mathrm{m} - \mathrm{r}}$}\\[12pt]
\textbf{性质4(二项式定理):}\\[8pt]
{\large $\sum_{i = 0}^{\mathrm{n}} \left( \mathrm{C}_{\mathrm{n}}^{i} \cdot x^{i} \right) = (1 + x)^{\mathrm{n}}$}\\[12pt]
{\large $\sum_{i = 0}^{\mathrm{n}} \mathrm{C}_{\mathrm{n}}^{i} = 2^{\mathrm{n}}$}\\[12pt]
\textbf{性质5:}\\[8pt]
{\large $\sum_{i = 0}^{\mathrm{n}} \left( (-1)^{i} \cdot \mathrm{C}_{\mathrm{n}}^{i} \right) = 0$}\\[12pt]
\textbf{性质6:}\\[8pt]
{\large $\mathrm{C}_{\mathrm{n}}^{0} + \mathrm{C}_{\mathrm{n}}^{2} + \cdots = \mathrm{C}_{\mathrm{n}}^{1} + \mathrm{C}_{\mathrm{n}}^{3} + \cdots = 2^{\mathrm{n} - 1}$}\\[12pt]
\textbf{性质7:}\\[8pt]
{\large $\mathrm{C}_{\mathrm{n} + \mathrm{m}}^{\mathrm{r}} = \sum_{i = 0}^{\min(\mathrm{n}, \mathrm{m}, \mathrm{r})} \left( \mathrm{C}_{\mathrm{n}}^{i} \cdot \mathrm{C}_{\mathrm{m}}^{\mathrm{r} - i} \right)$}\\[12pt]
{\large $\mathrm{C}_{\mathrm{n} + \mathrm{m}}^{\mathrm{n}} = \mathrm{C}_{\mathrm{n} + \mathrm{m}}^{\mathrm{m}} = \sum_{i = 0}^{\min(\mathrm{n}, \mathrm{m})} \left( \mathrm{C}_{\mathrm{n}}^{i} \cdot \mathrm{C}_{\mathrm{m}}^{i} \right), \quad (\mathrm{r} = \mathrm{n}\ |\ \mathrm{r} = \mathrm{m})$}\\[12pt]
\textbf{性质8:}\\[8pt]
{\large $\mathrm{m} \cdot \mathrm{C}_{\mathrm{n}}^{\mathrm{m}} = \mathrm{n} \cdot \mathrm{C}_{\mathrm{n} - 1}^{\mathrm{m} - 1}$}\\[12pt]
\textbf{性质9:}\\[8pt]
{\large $\sum_{i = 0}^{\mathrm{n}} \left( \mathrm{C}_{\mathrm{n}}^{i} \cdot i^2 \right) = \mathrm{n}(\mathrm{n} + 1) \cdot 2^{\mathrm{n} - 2}$}\\[12pt]
\textbf{性质10:}\\[8pt]
{\large $\sum_{i = 0}^{\mathrm{n}} \left( \mathrm{C}_{\mathrm{n}}^{i} \right)^2 = \mathrm{C}_{2\mathrm{n}}^{\mathrm{n}}$}\\[12pt]


\section{常数表}
\begin{center}
    \begin{tabular}{|r|r|r|r|r|r|}  
        \hline
        \rowcolor{gray!20}
        $n$         & $\log_{10}n$  & $n!$          & $C(n,n/2)$    & $\mathrm{LCM}(1...n)$ & $P_n$         \\ \hline  
        2           & 0.30102999    & 2             & 2             & 2                     & 2             \\ \hline 
        3           & 0.47712125    & 6             & 3             & 6                     & 3             \\ \hline 
        4           & 0.60205999    & 24            & 6             & 12                    & 5             \\ \hline 
        5           & 0.69897000    & 120           & 10            & 60                    & 7             \\ \hline 
        6           & 0.77815125    & 720           & 20            & 60                    & 11            \\ \hline 
        7           & 0.84509804    & 5040          & 35            & 420                   & 15            \\ \hline 
        8           & 0.90308998    & 40320         & 70            & 840                   & 22            \\ \hline 
        9           & 0.95424251    & 362880        & 126           & 2520                  & 30            \\ \hline 
        10          & 1.00000000    & 3628800       & 252           & 2520                  & 42            \\ \hline 
        11          & 1.04139269    & 39916800      & 462           & 27720                 & 56            \\ \hline 
        12          & 1.07918125    & 479001600     & 924           & 27720                 & 77            \\ \hline 
        15          & 1.17609126    & 1.31e12       & 6435          & 360360                & 176           \\ \hline 
        20          & 1.30103000    & 2.43e18       & 184756        & 232792560             & 627           \\ \hline 
        25          & 1.39794001    & 1.55e25       & 5200300       & 26771144400           & 1958          \\ \hline 
        30          & 1.47712125    & 2.65e32       & 155117520     & 1.444e14              & 5604          \\ \hline 
        $P_n$       & $37338_{40}$  & $204226_{50}$ & $966467_{60}$ & $190569292_{100}$     & $1e9_{114}$   \\ \hline 
    \end{tabular}\\
\end{center}
\begin{center}
    $\max\omega(n)$:小于等于n中的数最大质因数个数\\
    $\max d(n)$:小于等于n中的数最大因数个数\\
    $\pi(n)$:小于等于n中的数最大互质数个数\\
    \begin{tabular}{|r|r|r|r|r|r|r|}  
        \hline
        \rowcolor{gray!20}
        $n\leq$         & 10        & 100       & 1e3       & 1e4       & 1e5       & 1e6           \\ \hline  
        $\max\omega(n)$ & 2         & 3         & 4         & 5         & 6         & 7             \\ \hline 
        $\max d(n)$     & 4         & 12        & 32        & 64        & 128       & 240           \\ \hline 
        $\pi(n)$        & 4         & 25        & 168       & 1229      & 9592      & 78498         \\ \hline 
        \rowcolor{gray!20}
        $n\leq$         & 1e7       & 1e8       & 1e9       & 1e10      & 1e11      & 1e12          \\ \hline 
        $\max\omega(n)$ & 8         & 8         & 9         & 10        & 10        & 11            \\ \hline 
        $\max d(n)$     & 448       & 768       & 1344      & 2304      & 4032      & 6720          \\ \hline 
        $\pi(n)$        & 664579    & 5761455   & 5.08e7    & 4.55e8    & 4.12e9    & 3.7e10        \\ \hline 
        \rowcolor{gray!20}
        $n\leq$         & 1e13      & 1e14      & 1e15      & 1e16      & 1e17      & 1e18          \\ \hline 
        $\max\omega(n)$ & 12        & 12        & 13        & 13        & 14        & 15            \\ \hline 
        $\max d(n)$     & 10752     & 17280     & 26880     & 41472     & 64512     & 103680        \\ \hline 
        $\pi(n)$        & \multicolumn{5}{l}{Prime number theorem: $\pi(x) \sim \frac{x}{\log(x)}$}&\\ \hline 
    \end{tabular}
\end{center}