\documentclass{article}
\usepackage[a4paper, landscape, top=1cm, bottom=1.5cm, left=1.5cm, right=1.5cm]{geometry} % 设置页边距
\usepackage[fontset=none]{ctex}
\usepackage{titlesec}
\usepackage{listings}
\usepackage{xcolor}
\usepackage{fontspec}
\usepackage{multicol}

% 设置中文字体
\setCJKmainfont[AutoFakeBold={2.5}]{NSimSun} % 华文中宋
\setmonofont{Fira Code}

\linespread{1.6}
\setlength{\parskip}{0.5em}
\setlength{\parindent}{2em}
\setlength\columnsep{0.7cm} 

\titleformat{\section}
{\zihao{-3}\Large\bfseries}
{\thesection}{1em}{}

\titleformat{\subsection}
{\zihao{-3}\large\bfseries}
{\thesubsection}{0.5em}{}

% 定义代码高亮样式
\definecolor{codekeyword}{RGB}{0, 0, 136}
\definecolor{codetype}{RGB}{146, 0, 146}
\definecolor{codestring}{RGB}{0, 136, 0}
\definecolor{codepunctuality}{RGB}{146, 146, 0}
\definecolor{codenumber}{RGB}{0, 146, 146}
\definecolor{codecomment}{RGB}{136, 0, 0}

\lstdefinestyle{cppstyle}{
    language=C++,
    basicstyle=\ttfamily\footnotesize,
    breakatwhitespace=false,
    breaklines=true,
    keepspaces=true,
    numbers=left,
    numbersep=5pt,
    showspaces=false,
    showstringspaces=false,
    showtabs=false,
    tabsize=4,
    frame=single,
    backgroundcolor=\color{white},
    rulecolor=\color{black},
    commentstyle=\color{codecomment}\bfseries,
    stringstyle=\color{codestring}\bfseries,
    morekeywords={},
    morekeywords={
        if, while, else, using, for, do, false, ture, nullptr, struct, class, 
        public, protected, private, const, constexpr, return, template, assert
    },
    emph={
        int, char, float, double, bool, short, long, unsigned, signed, void, size_t, decltype, auto, 
        array, vector, list, map, set, unordered_map, unordered_set, stack, queue, priority_queue, pair, tuple, typename
    },
    keywordstyle=\color{codekeyword}\bfseries,
    emphstyle=\color{codetype}\bfseries
}

\begin{document}
\begin{multicols*}{2}

\thispagestyle{empty}
\tableofcontents
\newpage
\setcounter{page}{1}
\section{基础算法}
\subsection{三分}
\lstinputlisting[style=cppstyle]{./src/1 基础算法/三分.cpp}
\subsection{二分}
\lstinputlisting[style=cppstyle]{./src/1 基础算法/二分.cpp}
\subsection{快速幂}
\lstinputlisting[style=cppstyle]{./src/1 基础算法/快速幂.cpp}
\subsection{离散化}
\lstinputlisting[style=cppstyle]{./src/1 基础算法/离散化.cpp}
\section{图论}
\subsection{Tarjan割点}
\lstinputlisting[style=cppstyle]{./src/2 图论/Tarjan割点.cpp}
\subsection{Tarjan割边}
\lstinputlisting[style=cppstyle]{./src/2 图论/Tarjan割边.cpp}
\subsection{Tarjan强连通分量}
\lstinputlisting[style=cppstyle]{./src/2 图论/Tarjan强连通分量.cpp}
\subsection{Tarjan点双连通分量}
\lstinputlisting[style=cppstyle]{./src/2 图论/Tarjan点双连通分量.cpp}
\subsection{Tarjan边双连通分量}
\lstinputlisting[style=cppstyle]{./src/2 图论/Tarjan边双连通分量.cpp}
\subsection{拓扑排序}
\lstinputlisting[style=cppstyle]{./src/2 图论/拓扑排序.cpp}
\subsection{最小生成树kruskal}
\lstinputlisting[style=cppstyle]{./src/2 图论/最小生成树kruskal.cpp}
\subsection{最小生成树prim}
\lstinputlisting[style=cppstyle]{./src/2 图论/最小生成树prim.cpp}
\section{数据结构}
\subsection{Splay}
\lstinputlisting[style=cppstyle]{./src/3 数据结构/Splay.cpp}
\subsection{ST表}
\lstinputlisting[style=cppstyle]{./src/3 数据结构/ST表.cpp}
\subsection{对顶堆}
\lstinputlisting[style=cppstyle]{./src/3 数据结构/对顶堆.cpp}
\subsection{并查集}
\lstinputlisting[style=cppstyle]{./src/3 数据结构/并查集.cpp}
\subsection{树状数组}
\lstinputlisting[style=cppstyle]{./src/3 数据结构/树状数组.cpp}
\subsection{线段树}
\lstinputlisting[style=cppstyle]{./src/3 数据结构/线段树.cpp}
\section{树算法}
\subsection{树剖LCA}
\lstinputlisting[style=cppstyle]{./src/4 树算法/树剖LCA.cpp}
\section{数论}
\subsection{欧拉筛}
\lstinputlisting[style=cppstyle]{./src/5 数论/欧拉筛.cpp}
\section{字符串}
\subsection{EXKMP}
\lstinputlisting[style=cppstyle]{./src/6 字符串/EXKMP.cpp}
\subsection{KMP}
\lstinputlisting[style=cppstyle]{./src/6 字符串/KMP.cpp}
\subsection{字符串哈希}
\lstinputlisting[style=cppstyle]{./src/6 字符串/字符串哈希.cpp}
\subsection{马拉车}
\lstinputlisting[style=cppstyle]{./src/6 字符串/马拉车.cpp}
\end{multicols*}
\end{document}
