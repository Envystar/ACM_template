\documentclass{article}
\usepackage[a4paper, landscape, top=1cm, bottom=1.5cm, left=1cm, right=1cm]{geometry} % 设置页边距
\usepackage[fontset=none]{ctex}
\usepackage{titlesec}
\usepackage{listings}
\usepackage{xcolor}
\usepackage{fontspec}
\usepackage{multicol}
\usepackage{hyperref} 

\hypersetup{
  colorlinks=true,
  linkcolor=black,
  pdfstartview=Fit,
  breaklinks=true,
  linktoc=all
}

% 设置中文字体
\setCJKmainfont[AutoFakeBold={2.5}]{NSimSun} % 华文中宋
\setmonofont{Fira Code}

\linespread{1.6}
\setlength{\parskip}{0.5em}
\setlength{\parindent}{0em}
\setlength\columnsep{0.4cm} 

\titleformat{\section}
{\zihao{-3}\Large\bfseries}
{\thesection}{1em}{}

\titleformat{\subsection}
{\zihao{-3}\large\bfseries}
{\thesubsection}{0.5em}{}

% 定义代码高亮样式
\definecolor{codekeyword}{RGB}{0, 0, 136}
\definecolor{codetype}{RGB}{146, 0, 146}
\definecolor{codestring}{RGB}{0, 136, 0}
\definecolor{codepunctuality}{RGB}{146, 146, 0}
\definecolor{codenumber}{RGB}{0, 146, 146}
\definecolor{codecomment}{RGB}{136, 0, 0}

\lstdefinestyle{cppstyle}{
    language=C++,
    basicstyle=\ttfamily\tiny,
    breakatwhitespace=false,
    breaklines=true,
    keepspaces=true,
    %numbers=left,
    %numbersep=5pt,
    showspaces=false,
    showstringspaces=false,
    showtabs=false,
    tabsize=4,
    frame=single,
    backgroundcolor=\color{white},
    rulecolor=\color{black},
    commentstyle=\color{codecomment}\bfseries,
    stringstyle=\color{codestring}\bfseries,
    morekeywords={},
    morekeywords={
        if, while, else, using, for, do, false, ture, nullptr, struct, class, 
        public, protected, private, const, constexpr, return, template, assert
    },
    emph={
        int, char, float, double, bool, short, long, unsigned, signed, void, size_t, decltype, auto, int64_t, 
        array, vector, list, map, set, unordered_map, unordered_set, stack, queue, priority_queue, pair, tuple, typename
    },
    keywordstyle=\color{codekeyword}\bfseries,
    emphstyle=\color{codetype}\bfseries
}

\begin{document}
\begin{multicols*}{3}

\thispagestyle{empty}
\tableofcontents
\newpage
\setcounter{page}{1}
\section{基础算法}
\subsection{三分}
\input{./src/1 基础算法/三分.tex}
\lstinputlisting[style=cppstyle]{./src/1 基础算法/三分.cpp}
\section{图论}
\subsection{图的连通性}
\subsubsection{Tarjan割点}
\input{./src/2 图论/图的连通性/Tarjan割点.tex}
\lstinputlisting[style=cppstyle]{./src/2 图论/图的连通性/Tarjan割点.cpp}
\subsubsection{Tarjan割边}
\input{./src/2 图论/图的连通性/Tarjan割边.tex}
\lstinputlisting[style=cppstyle]{./src/2 图论/图的连通性/Tarjan割边.cpp}
\subsubsection{Tarjan强连通分量}
\input{./src/2 图论/图的连通性/Tarjan强连通分量.tex}
\lstinputlisting[style=cppstyle]{./src/2 图论/图的连通性/Tarjan强连通分量.cpp}
\subsubsection{Tarjan点双连通分量}
\input{./src/2 图论/图的连通性/Tarjan点双连通分量.tex}
\lstinputlisting[style=cppstyle]{./src/2 图论/图的连通性/Tarjan点双连通分量.cpp}
\subsubsection{Tarjan边双连通分量}
\tiny {
tarjan求边双连通分量

Link: https://www.luogu.com.cn/problem/P8436
}
\lstinputlisting[style=cppstyle]{./src/2 图论/图的连通性/Tarjan边双连通分量.cpp}
\subsection{拓扑排序}
\input{./src/2 图论/拓扑排序.tex}
\lstinputlisting[style=cppstyle]{./src/2 图论/拓扑排序.cpp}
\subsection{最小生成树}
\subsubsection{Kruskal}
\input{./src/2 图论/最小生成树/Kruskal.tex}
\lstinputlisting[style=cppstyle]{./src/2 图论/最小生成树/Kruskal.cpp}
\subsubsection{Prim}
\input{./src/2 图论/最小生成树/Prim.tex}
\lstinputlisting[style=cppstyle]{./src/2 图论/最小生成树/Prim.cpp}
\subsection{树的重心}
\tiny{
如果在树中选择某个节点并删除,这棵树将分为若干棵子树,统计子树节点数并记录最大值。取遍树上所有节点,使此最大值取到最小的节点被称为整个树的重心。
}
\lstinputlisting[style=cppstyle]{./src/2 图论/树的重心.cpp}
\subsection{流和匹配}
\subsubsection{EdmondsKarp}
\textbf{时间复杂度:} $O(|V||E|^2)$ 实际情况一般远低于此复杂度\\
\textbf{空间复杂度:} $O(|V| + |E|)$ \\
\textbf{用途:}求最大流\\
\textbf{模板题:}\href{https://www.luogu.com.cn/problem/P3376}{洛谷 P3376}

\lstinputlisting[style=cppstyle]{./src/2 图论/流和匹配/EdmondsKarp.cpp}
\section{数据结构}
\subsection{Splay}
\input{./src/3 数据结构/Splay.tex}
\lstinputlisting[style=cppstyle]{./src/3 数据结构/Splay.cpp}
\subsection{ST表}
\tiny {
\textbf{时间复杂度:}\\
\textbullet{初始化:$O(n\log(n))$}\\
\textbullet{查询:$O(1)$}\\
\textbf{空间复杂度:}$O(nlog(n))$\\
\textbf{用途:}RMQ问题,不支持修改\\
\textbf{模板题:}\href{https://www.luogu.com.cn/problem/P3865}{Luogu P3865}
}
\lstinputlisting[style=cppstyle]{./src/3 数据结构/ST表.cpp}
\subsection{对顶堆}
\input{./src/3 数据结构/对顶堆.tex}
\lstinputlisting[style=cppstyle]{./src/3 数据结构/对顶堆.cpp}
\subsection{并查集}
\input{./src/3 数据结构/并查集.tex}
\lstinputlisting[style=cppstyle]{./src/3 数据结构/并查集.cpp}
\subsection{树状数组}
\subsubsection{树状数组}
\input{./src/3 数据结构/树状数组/树状数组.tex}
\lstinputlisting[style=cppstyle]{./src/3 数据结构/树状数组/树状数组.cpp}
\subsubsection{树状数组2}
\input{./src/3 数据结构/树状数组/树状数组2.tex}
\lstinputlisting[style=cppstyle]{./src/3 数据结构/树状数组/树状数组2.cpp}
\subsection{波纹疾走树}
\input{./src/3 数据结构/波纹疾走树.tex}
\lstinputlisting[style=cppstyle]{./src/3 数据结构/波纹疾走树.cpp}
\subsection{线段树}
\subsubsection{主席树}
\input{./src/3 数据结构/线段树/主席树.tex}
\lstinputlisting[style=cppstyle]{./src/3 数据结构/线段树/主席树.cpp}
\subsubsection{标记永久化主席树}
\input{./src/3 数据结构/线段树/标记永久化主席树.tex}
\lstinputlisting[style=cppstyle]{./src/3 数据结构/线段树/标记永久化主席树.cpp}
\subsubsection{线段树}
\input{./src/3 数据结构/线段树/线段树.tex}
\lstinputlisting[style=cppstyle]{./src/3 数据结构/线段树/线段树.cpp}
\subsubsection{线段树优化建图}
\input{./src/3 数据结构/线段树/线段树优化建图.tex}
\lstinputlisting[style=cppstyle]{./src/3 数据结构/线段树/线段树优化建图.cpp}
\subsection{重链剖分}
\input{./src/3 数据结构/重链剖分.tex}
\lstinputlisting[style=cppstyle]{./src/3 数据结构/重链剖分.cpp}
\section{数学}
\subsection{数论}
\subsubsection{MillerRabin}
\textbf{模板题:}\href{https://www.luogu.com.cn/problem/SP288}{Luogu SP288}\\
\textbf{时间复杂度:} $O(k \log^3(n)), k = 7$
\lstinputlisting[style=cppstyle]{./src/4 数学/数论/MillerRabin.cpp}
\subsubsection{PollardRho}
\textbf{模板题:}\href{https://judge.yosupo.jp/problem/factorize}{Factorize}\\
\textbf{时间复杂度:} $O(n ^ {\frac{1}{4}})$

\lstinputlisting[style=cppstyle]{./src/4 数学/数论/PollardRho.cpp}
\subsubsection{区间筛}
\input{./src/4 数学/数论/区间筛.tex}
\lstinputlisting[style=cppstyle]{./src/4 数学/数论/区间筛.cpp}
\subsubsection{欧拉筛}
\input{./src/4 数学/数论/欧拉筛.tex}
\lstinputlisting[style=cppstyle]{./src/4 数学/数论/欧拉筛.cpp}
\subsection{组合数学}
\subsubsection{卢卡斯定理}
\tiny {
$C_n^m \pmod p = C_{\lfloor \frac{n}{p} \rfloor}^{\lfloor \frac{m}{p} \rfloor} * C_{n \bmod p}^{m \bmod p}$ 
}

\lstinputlisting[style=cppstyle]{./src/4 数学/组合数学/卢卡斯定理.cpp}
\section{字符串}
\subsection{EXKMP}
\input{./src/5 字符串/EXKMP.tex}
\lstinputlisting[style=cppstyle]{./src/5 字符串/EXKMP.cpp}
\subsection{KMP}
\input{./src/5 字符串/KMP.tex}
\lstinputlisting[style=cppstyle]{./src/5 字符串/KMP.cpp}
\subsection{字符串哈希}
\input{./src/5 字符串/字符串哈希.tex}
\lstinputlisting[style=cppstyle]{./src/5 字符串/字符串哈希.cpp}
\subsection{马拉车}
\input{./src/5 字符串/马拉车.tex}
\lstinputlisting[style=cppstyle]{./src/5 字符串/马拉车.cpp}
\section{计算几何}
\subsection{凸包}
\input{./src/6 计算几何/凸包.tex}
\lstinputlisting[style=cppstyle]{./src/6 计算几何/凸包.cpp}
\section{杂项}
\subsection{康托展开}
\input{./src/7 杂项/康托展开.tex}
\lstinputlisting[style=cppstyle]{./src/7 杂项/康托展开.cpp}
\subsection{逆康托展开}
\input{./src/7 杂项/逆康托展开.tex}
\lstinputlisting[style=cppstyle]{./src/7 杂项/逆康托展开.cpp}
\subsection{高精度}
\input{./src/7 杂项/高精度.tex}
\lstinputlisting[style=cppstyle]{./src/7 杂项/高精度.cpp}
\subsection{高维前缀和}
\tiny {
\textbf{时间复杂度:}$O(n2^n)$\\
\textbf{空间复杂度:}$O(n2^n)$\\
\textbf{用途:}位集合中,求出某个集合的所有子集值之和以及其他可加性操作\\
\textbf{模板题:}\href{https://atcoder.jp/contests/arc100/tasks/arc100_c}{AtCoder ARC100 C}
}
\lstinputlisting[style=cppstyle]{./src/7 杂项/高维前缀和.cpp}
\end{multicols*}
\end{document}
